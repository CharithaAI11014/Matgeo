\documentclass{beamer}
\mode<presentation>
\usepackage{amsmath}
\usepackage{amssymb}
\usepackage{adjustbox}
\usepackage{subcaption}
\usepackage{enumitem}
\usepackage{multicol}
\usepackage{mathtools}
\usepackage{listings}
\usepackage{hyperref}
\usepackage{url}
\def\UrlBreaks{\do\/\do-}
\usetheme{Boadilla}
\usecolortheme{lily}
\setbeamertemplate{footline}
{
  \leavevmode%
  \hbox{%
  \begin{beamercolorbox}[wd=\paperwidth,ht=5.25ex,dp=1ex,right]{author in head/foot}%
    \insertframenumber{} / \inserttotalframenumber\hspace*{2ex} 
  \end{beamercolorbox}}%
  \vskip0pt%
}
\setbeamertemplate{navigation symbols}{}

\providecommand{\nCr}[2]{\,^{#1}C_{#2}} % nCr
\providecommand{\nPr}[2]{\,^{#1}P_{#2}} % nPr
\providecommand{\mbf}{\mathbf}
\providecommand{\pr}[1]{\ensuremath{\Pr\left(#1\right)}}
\providecommand{\qfunc}[1]{\ensuremath{Q\left(#1\right)}}
\providecommand{\sbrak}[1]{\ensuremath{{}\left[#1\right]}}
\providecommand{\lsbrak}[1]{\ensuremath{{}\left[#1\right.}}
\providecommand{\rsbrak}[1]{\ensuremath{{}\left.#1\right]}}
\providecommand{\brak}[1]{\ensuremath{\left(#1\right)}}
\providecommand{\lbrak}[1]{\ensuremath{\left(#1\right.}}
\providecommand{\rbrak}[1]{\ensuremath{\left.#1\right)}}
\providecommand{\cbrak}[1]{\ensuremath{\left\{#1\right\}}}
\providecommand{\lcbrak}[1]{\ensuremath{\left\{#1\right.}}
\providecommand{\rcbrak}[1]{\ensuremath{\left.#1\right\}}}
\theoremstyle{remark}
\newtheorem{rem}{Remark}
\newcommand{\sgn}{\mathop{\mathrm{sgn}}}
\providecommand{\abs}[1]{\left\vert#1\right\vert}
\providecommand{\res}[1]{\Res\displaylimits_{#1}} 
\providecommand{\norm}[1]{\lVert#1\rVert}
\providecommand{\mtx}[1]{\mathbf{#1}}
\providecommand{\mean}[1]{E\left[ #1 \right]}
\providecommand{\fourier}{\overset{\mathcal{F}}{ \rightleftharpoons}}
%\providecommand{\hilbert}{\overset{\mathcal{H}}{ \rightleftharpoons}}
\providecommand{\system}{\overset{\mathcal{H}}{ \longleftrightarrow}}
\providecommand{\dec}[2]{\ensuremath{\overset{#1}{\underset{#2}{\gtrless}}}}
\newcommand{\myvec}[1]{\ensuremath{\begin{pmatrix}#1\end{pmatrix}}}
\let\vec\mathbf

\lstset{
%language=C,
frame=single, 
breaklines=true,
columns=fullflexible
}

\numberwithin{equation}{section}
\title{Using Matrices to Find the External Division Point of a Line Segment}
\author{Charitha Sri \\ AI24BTECH11014}
\date{\today}

\begin{document}

\begin{frame}
    \titlepage
\end{frame}

\begin{frame}
    \frametitle{Problem Statement}
    Given points $\vec{X}$ and $\vec{Y}$ with position vectors:
    \[
    \vec{X} = 3\vec{a} + \vec{b}, \quad \vec{Y} = \vec{a} - 3\vec{b}
    \]
    find the position vector of point $\vec{V}$ which divides the line segment $XY$ in the ratio $2 : 1$ externally.
\end{frame}

\begin{frame}
    \frametitle{Section Formula for External Division}
    If a point $\vec{V}$ divides the line segment joining $\vec{X}$ and $\vec{Y}$ externally in the ratio $m : n$, then:
    \[
    \vec{V} = \frac{m\vec{Y} - n\vec{X}}{m - n}
    \]
    Here, $m = 2$ and $n = 1$. This approach is based on the section formula, which is commonly used in analytic geometry.
\end{frame}

\begin{frame}
    \frametitle{Why Use Matrices?}
    \begin{itemize}
        \item \textbf{Efficiency in Computation}: Using matrices allows us to represent and manipulate vectors efficiently, especially when working with multiple points or vector transformations.
        \item \textbf{Code Simplification}: Matrix representations make coding vector calculations easier, reducing the risk of errors when handling individual coordinates.
        \item \textbf{Scalability}: The matrix approach can easily be extended to higher dimensions or more complex transformations, which is useful in fields such as computer graphics and machine learning.
    \end{itemize}
\end{frame}

\begin{frame}
    \frametitle{Calculating $\vec{V}$ Using the Section Formula}
    Substitute $\vec{X} = 3\vec{a} + \vec{b}$ and $\vec{Y} = \vec{a} - 3\vec{b}$ into the formula:
    \[
    \vec{V} = \frac{2\myvec{ 1 && -3} \myvec{\vec{a} \\ \vec{b}} - 1 \myvec{3 && 1} \myvec{\vec{a} \\ \vec{b}}}{2 - 1}
    \]
    Simplifying,
    \begin{align}
    \label{eq: Position vector on X and Y}
        \vec{V} &= -\vec{a} - 7\vec{b}
    \end{align}
\end{frame}

\begin{frame}
    \frametitle{Results and Visualization}
    \begin{listings} \url{https://github.com/CharithaAI11014/Matgeo/blob/master/codes/plot.py} 
    \end{listings} plots the position vector \eqref{eq: Position vector on X and Y}

    The code in 
    \begin{listings} \url{https://github.com/CharithaAI11014/Matgeo/blob/master/codes/section.c} 
    \end{listings} 
    uses matrices and Matsecext function and calculates the position vector.
\end{frame}

\end{document}

