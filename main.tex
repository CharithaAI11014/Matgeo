\let\negmedspace\undefined
\let\negthickspace\undefined
\documentclass[journal]{IEEEtran}
\usepackage[a5paper, margin=10mm, onecolumn]{geometry}
%\usepackage{lmodern} % Ensure lmodern is loaded for pdflatex
\usepackage{tfrupee} % Include tfrupee package

\setlength{\headheight}{1cm} % Set the height of the header box
\setlength{\headsep}{0mm}     % Set the distance between the header box and the top of the text

\usepackage{gvv-book}
\usepackage{gvv}
\usepackage{cite}
\usepackage{amsmath,amssymb,amsfonts,amsthm}
\usepackage{algorithmic}
\usepackage{graphicx}
\usepackage{textcomp}
\usepackage{xcolor}
\usepackage{txfonts}
\usepackage{listings}
\usepackage{enumitem}
\usepackage{mathtools}
\usepackage{gensymb}
\usepackage{comment}
\usepackage[breaklinks=true]{hyperref}
\usepackage{tkz-euclide} 
\usepackage{listings}
% \usepackage{gvv}                                        
\def\inputGnumericTable{}                                 
\usepackage[latin1]{inputenc}                                
\usepackage{color}                                            
\usepackage{array}                                            
\usepackage{longtable}                                       
\usepackage{calc}                                             
\usepackage{multirow}                                         
\usepackage{hhline}                                           
\usepackage{ifthen}                                           
\usepackage{lscape}


\renewcommand{\thefigure}{\theenumi}
\renewcommand{\thetable}{\theenumi}
\setlength{\intextsep}{10pt} % Space between text and floats

\numberwithin{equation}{enumi}
\numberwithin{figure}{enumi}
\renewcommand{\thetable}{\theenumi}	

% Marks the beginning of the document
\begin{document}
\bibliographystyle{IEEEtran}

\title{1-1.5-22}
\author{AI24BTECH11014 \\ Charitha Sri}

% \maketitle
% \newpage
% \bigskip
{\let\newpage\relax\maketitle}

\textbf{QUESTION} \\$\vec{X}$ and $\vec{Y}$ are two points with position vectors $3\overrightarrow{a}+ \overrightarrow{b}$ and $\overrightarrow{a}-3\overrightarrow{b}$ respectively. Write
the position vector of a point $\vec{V}$ which divides the line segment $XY$ in the ratio $2 : 1$ externally.\\
\textbf{Solution:} Given,\\
\begin{table}[h!]    
  \centering
\begin{tabular}[4pt]{| c| c| c|}
\hline
	\textbf{Variable} & \textbf{Description} & \textbf{Formula} \\
\hline
	point $\vec{X}$ & $\myvec{3&&1}\myvec{\vec{a} \\ \vec{b}}$ & - \\
\hline 
	point $\vec{Y}$ & $\myvec{1&&-3}\myvec{\vec{a} \\ \vec{b}}$ &  $\vec{Y} = \frac{\vec{V}+n\vec{X}}{1+n}$\\
\hline
	Ratio of $\frac{VX}{VY}$ & $\frac{2}{1}$ & - \\
\hline
	point $\vec{V}$ & Point on line $XY$ & - \\
\hline
\end{tabular}

  \label{table: 1-1.5-22}
  \end{table}\\
 As, the point $\vec{V}$ divides the line $XY$ externally,
\begin{align}
XY = VX - VY \\
\frac{VX}{VY}=\frac{2}{1}\\
\frac{VY}{YX}=\frac{1}{1}\\
 n = 1 \\
\end{align}
$\vec{Y}$ divides the line joining the points $V$ and $X$ internally in the ratio $n : 1$ \\
By section formula, $\vec{Y}$ can be expressed as\\ 
\begin{align}
\vec{Y}=\frac{1}{2}\brak{\vec{V}+\vec{X}}\\
 \vec{V} = 2\myvec{ 1 && -3}\myvec{ \vec{a} \\ \vec{b}} - \myvec{3 && 1}\myvec{\vec{a} \\ \vec{b}}\\
 \vec{V}= \myvec{-1 && -7}\myvec{\vec{a} \\ \vec{b}} \\
\end{align}
 Therefore the position vector of point $\vec{V}$ is $-\overrightarrow{a}-7\overrightarrow{b}$
\begin{figure}[ht]
        \centering
        \includegraphics[width=0.7\linewidth]{figs/fig.png}
        \caption{}
        \label{graph}
\end{figure}

\end{document}
